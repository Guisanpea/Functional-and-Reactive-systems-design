\documentclass[../main.tex]{subfiles}

\begin{document}
\begin{abstract}

As today's user demands faster changes on software systems, a programming foundation based on easy to change programs is needed. Functional Programming constructs are highly composable and free of side effects. For this reason, this paper studies it as a paradigm for creating programs that can adapt to new requirements in a fast manner. With application responsiveness being to an increasingly important trait of user-facing applications, Reactive Systems are studied as a frame of reference for creating applications that respond fast to user interactions and are fault-tolerant within the functional scope.

Functional and Reactive systems concepts are not only presented on a theoretical basis but are also accompanied by examples and common patterns that make the most profit from them.

Finally, a functional and reactive web application that makes use of all these concepts is developed putting in practice all the introduced concepts and showcasing a practical application that uses persistent storage and network communication in a purely functional manner using the reactive principles.

\bfseries{\large{Keywords:} Functional Programming, Scala, Reactive Systems}

\end{abstract}
\end{document}