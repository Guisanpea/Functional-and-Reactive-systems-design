\documentclass[../main.tex]{subfiles}

\begin{document}
%% begin abstract format
\makeatletter
\renewenvironment{abstract}{%
    \if@twocolumn
      \section*{Resumen \\}%
    \else %% <- here I've removed \small
    \begin{flushright}
        {\filleft\Huge\bfseries\fontsize{48pt}{12}\selectfont Resumen\vspace{\z@}}%  %% <- here I've added the format
        \end{flushright}
      \quotation
    \fi}
    {\if@twocolumn\else\endquotation\fi}
\makeatother
%% end abstract format
\begin{abstract}
Dado que los usuarios hoy en día requieren actualizaciones constantes del software que usan, 
son necesarios unos fundamentos de programación que permitan realizar cambios frecuentes 
en estos sistemas de una manera lo más simple posible. Los elementos propios de la Programación Funcional 
son fáciles de componer y están libres de efectos colaterales. Es por ello que este documento 
estudia la programación funcional como un paradigma para crear programas que pueden adaptarse 
con facilidad a los cambios que implican la evolución de requisitos. Al mismo tiempo, 
la capacidad del software de responder rápidamente a las interacciones con los usuarios 
es cada vez más importante. Los Sistemas Reactivos establecen un marco de referencia 
para crear aplicaciones que sean capaces de interaccionar con un número elevado de usuarios
con una baja latencia y que sean además tolerantes a fallos.

Estos conceptos no son presentados exclusivamente de un punto de vista teórico sino 
que también se acompañan de ejemplos prácticos y patrones útiles para crear Sistemas 
Funcionales y Reactivos.

Finalmente, se estudia un caso de uso práctico en el que se utilizan todos los elementos 
que se han estudiado, implementando una aplicación que utiliza almacenamiento persistente 
y comunicación a través del protocolo HTTP de una forma puramente funcional, siguiendo además
los principios propios de la programación reactiva.

\bfseries{\large{Palabras Clave:} Programación Funcional, Scala, Sistemas Reactivos}

\end{abstract}
\end{document}


