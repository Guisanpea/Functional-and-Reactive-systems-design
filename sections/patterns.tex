%%% Local Variables:
%%% mode: latex
%%% TeX-master: "main"
%%% End:

\documentclass[../main.tex]{subfiles}

\begin{document}

\subsection{Improving Resilience with Computational effects}
In 1991 Eugenio Moggi [[Insert citation]] described a model in category theory
for separating the set of values a object \texttt{A} could have from the notion
of a computation of it, \texttt{T A}, which had an effect on top of this type.
This computation abstracts from the type \texttt{A} the posible results that the
computation could have. Some examples of computations mentioned on the text of
interest to this topics are.

\begin{itemize}
\item Side effects that modify an set of possible states \texttt{S}: \texttt{State[A, S]}
\item Exceptions where \texttt{E} is the set of possible exceptions: \texttt{Either[A, E]}
\item Interactive Input/Output which results in an \texttt{A} value: \texttt{IO[A]}
\end{itemize}

From a programming point of view a very interesting finding of that article is
that this effect over \texttt{A} can be abstracted and define operations that transform
the type A regardless of the effect in which it was contained as long as the
effect \texttt{F} is a monad. %% Is a monad? Has a monad?

\subsubsection{Monads}
A monad is category that provides two operations over a type constructor
\texttt{F[\_]}, \texttt{flatMap}\footnote{On literature is also usually called bind} and
\texttt{pure}\footnote{Also known as point or return}. In the following figure the a
trait that defines the Monad typeclass in Scala is defined.

[[Example]]

The semantics of the operations can be inferred by following the types of them.
The pure function takes a value \texttt{a} and lifts it to the monad context
\texttt{F}. The flatMap operation takes a monad \texttt{F[a]} and a function
which transform an \texttt{a} into a monad with the same context but a possible
different type \texttt{F[B]}. This operation mantains the monad context.

\subsection{Example of monadic effects}
Then some monads instances can be defined for different effects.

\paragraph{Option Monad}
The Option Monad is an effect which may have a value of a type \texttt{A} or
not. An \texttt{Option[A]} has two possible values. \texttt{Some} which means
that the value is present or \texttt{None} meaning that there is no value.

The function flatMap in this case returns a transformed \texttt{Some} if the monad in which
it was applied was also a \texttt{Some} or directly returns a \texttt{None} if
the monad was empty and no transformation could be applied

\paragraph{Either Monad}
When a function can either succeed or fail the Either Effect encodes this
possibility, represented with the type \texttt{Either[E, A]} it has two possible
disjoint values. Either are usually right biased, being the \texttt{A} or
``right'' value the one meaning the success and the \texttt{E} value the one
meaning an error.

\paragraph{State Monad}
The State Monad is an effect which given an state of type \texttt{S} performs an state
transition resulting in a value of type \texttt{A}, modeling an state machine.
As state cannot be mutated in place the state transation returns both the new
state and the resulting value. It can be defined this way

When using the state monad flatMap directly returns a monad with the new state
avoiding to explicitly pass the state all along the way.

\paragraph{IO Monad}
Side effects are inherently non functional. A side effect such as reading input
from the keyboard implies that every time that this operation is called the
outcome may be different breaking referential transparency.

However if we look to the description of operations that do side effects they
have nothing that implies a breaking of referential transparency. For example
the function read from keyboard whould have the type \texttt{() => A} which a
pure function could have.

If the side effect is abstracted from the function definition the purity is
mantained. This is what the IO monad does. When a function is passed to the IO
the execution is suspended. It is not executed until the programmer runs the
computation, moment at which purity is lost.

If side effects execution is delayed until the very end of the execution programs
can keep functional at its core and the benefits of purity are not lost until
the interpretation of the side effects.

\subsubsection{The resilience of computational effects}
In static typed programming languages computational effects offers a greater resilience as the
programmer has to explicitly deal with the effect and can not ignore the
effectful nature of it like having a possible error, an Either effect, or being a possibly
future value, i. e. IO.

This implicit treatment of effects like the use of a \texttt{Null} pointer has
lead to great number of mistakes, being called by one of the first designer of
it, Tony Hoare, its billion dolar mistake.

Nullability is not the only case of implicit effects. The use of unchecked
exceptions in object oriented programming languages can let errors be propagated
without the programmer supervision.

Effects are then a pattern that improve the reactiveness of functional programs by
improving the resilience with the use of the type system.

\subsection{Leveraging Multiprocessing by the use of Futures}
Futures, also called Promises, model an asynchronous computational model in
which operations can be ran in a separate thead of execution from the calling
thread of the future itself.

A Future can be created using the \texttt{apply} method of the companion object
of the trait \texttt{Future}. An invocation have this form
\texttt{Future(\textit{expr})}. This would evaluate expr in the background. The
call returns immediatly with a result of type Future[A], being A the result type
of the expression \texttt{\textit{expr}}.

Futures favour the use of non blocking computation. Computations that could
block the thread of execution, e.g. Blocking I/O like fetching a resource
through the network, should be ran inside Futures. This contributes to
the overall application reactiveness by terms of responsivenes by improving the
overall request latency and by reducing the load of the main execution thread.

\paragraph{Latency improvements of futures}

By spawning blocking calls in separate threads of execution the response time of
a request transitions from being the sum of the latencies of the independent
blocking operations to the latency of the longest one. This is represented in
figure \ref{fig:futurelatency}

\begin{figure}[ht]
  \centering
  \includegraphics[width=\textwidth1]{futurelatency.png}
  \caption{\label{fig:futurelatency}
    At the left of the figure a sequential run of the blocking operations. At
    the right a parallel execution which combines the results
  }
\end{figure}

\end{document}