%%% Local Variables:
%%% mode: latex
%%% TeX-master: "main"
%%% End:

\documentclass[../main.tex]{subfiles}

\begin{document}

\subsection{Abstracting over Computational Effects}
In 1991 Eugenio Moggi \autocite{Moggi1991NotionsMonads} described a model
in category theory for separating the set of values a object of type \texttt{A} may have from
the notion of computing those values, \texttt{T A}, which denotes an effect on top of
this type. This model abstracts from type \texttt{A} the possible
results that the computation could have. Some examples of computations mentioned
on that text of interest to this topics are:

\begin{itemize}
\item Side effects, that modify a set of possible states \texttt{S}: \texttt{State[A, S]}.
\item Exceptions, where \texttt{E} is the set of possible exceptions: \texttt{Either[A, E]}.
\item Interactive Input/Output, which results in a value of type \texttt{A}: \texttt{IO[A]}.
\end{itemize}

From a functional programming perspective, a very interesting finding of that
work is that thess effects over type \texttt{A} can be abstracted over and we can define
operations that transform the type A regardless of the effect in which it is
contained, as long as the effect \texttt{F} is a monad. %% Has a monad?
                                %% Is a monad?

\subsubsection{Monads}
A monad is category that provides two operations over a type constructor
\texttt{F[\_]}: \texttt{flatMap}, also usually called b\texttt{bind}, and \texttt{pure},
known as point or return. In Listing \ref{lst:monad}, a trait that defines the
Monad typeclass in Scala is shown.

\lstinputlisting[
caption=The monad trait.,
float=h,
captionpos=b,
label={lst:monad}
]{Monad.sc}

The semantics of the operations can be inferred from their types.
The pure function takes a value \texttt{a} and lifts it into the monad context
\texttt{F}. The flatMap operation takes a monad \texttt{F[a]} and a function,
which transforms an \texttt{a} into a monad with the same context but a with possible
different type inside, \texttt{F[B]}. This operation mantains the monad context.

\subsection{Example of monadic effects}
Some monad instances can be defined for representing different effects.

\paragraph{Option Monad}
The Option Monad is an effect which may represent a value of a type \texttt{A} or
not. An \texttt{Option[A]} has two possible values. \texttt{Some(a)}, which means
that the value is present, or \texttt{None}, meaning that there is no value.

The function flatMap in this case returns a transformed \texttt{Some} if the monad in which
it was applied was also a \texttt{Some} or directly returns a \texttt{None} if
the monad was empty and no transformation could be applied.

\lstinputlisting[
caption=The option monad instance.,
float=h,
captionpos=b,
label={lst:optionmonad}
]{OptionMonad.sc}

\paragraph{Either Monad}
When a function can either succeed or fail the Either Effect can be used to encode this
possibility. Represented with type \texttt{Either[E, A]}, it has two possible
disjoint values. Either are usually right biased, being the \texttt{A} or
``right'' value the one meaning the success and the \texttt{E} value the one
meaning an error.

\lstinputlisting[
caption=The either monad instance.,
float=h,
captionpos=b,
label={lst:eithermonad}
]{EitherMonad.sc}

\paragraph{State Monad}
The State Monad is an effect which, given a state of type \texttt{S}, performs a state
transition resulting in a value of type \texttt{A}, modeling an state machine.
As state cannot be mutated in place, the state transition returns both the new
state and the resulting value.

When using the state monad, flatMap directly returns a monad with the new state,
avoiding the need to explicitly pass the state all along the way, as shown in Listing \ref{lst:statemonad}.

\lstinputlisting[
caption=The state monad instance.,
float=h,
captionpos=b,
label={lst:statemonad}
]{StateMonad.sc}

\paragraph{IO Monad}
Side effects are inherently non functional. A side effect such as reading input
from the keyboard implies that every time that this operation is called the
outcome may be different, breaking referential transparency.

However, if we look to the description of operations that do side effects, they
have nothing that implies this lose of referential transparency. For example,
a function to read from keyboard whould have type \texttt{() => A} which is a
perfectly valid type for a pure function.

If the execution of the function with the side effect is delayed, what it
would end up being is just a description of it, without exposing its impure
nature. This is what the IO monad does. When a function is passed to the IO
the execution, is suspended and the description of such computation is captured. It is not
until the programmer runs the descriptions wrapped in the IO Monad that the
referential transparency is lost.

If this side effect execution is delayed until the very end of program execution, these operations
can keep functional at its core and the benefits of purity are not lost until
the interpretation of the side effects, which is usually done as the last step of
the computation. Usually, in order to avoid running the execution of the side
effects manually, the entry function of the application expects an IO to be
returned which will be then be interpreted. This is exactly what the main method
of Haskell does\autocite{Hudak2000AIO}.

Alternatively, the IO datatype can be thought of as an state transformer over 
the state not belonging to the program itself but to the external world. When
interpreted, this state is read through external ports, i.e.
network card, peripherals, storage disk, \ldots, and, at the same time, the new produced 
state can be ``written" to these external elements.

\subsubsection{The resilience of Computational Effects}
In statically typed programming languages, computational effects offer a greater resilience as the
programmer has to explicitly deal with the effect and can not ignore the
effectful nature of it, like having a possible error, an Either effect, or being a possibly
future value, i. e. an IO effect.

This implicit treatment of effects, just like the use of a \texttt{null} pointer, has
lead to great a number of mistakes, being called by Tony Hoare, one of the first designers of
programming languages, his billion dolar mistake \autocite{NullMistake}.

Nullability is not the only case of implicit effect in programming. The use of unchecked
exceptions in object oriented programming languages can let errors get propagated
without the programmer supervision.

Effects are then a pattern that improve the reactiveness of functional programs by
improving their resilience thanks to the use of type system.

\subsection{Leveraging Multiprocessing by means of Futures}
Futures, also called Promises, provide an asynchronous computational model in
which computations can be ran in a separate thead of execution, appart from the
thread calling the future itself.

A Future can be created by using the \texttt{apply} method of the companion object
of the trait \texttt{Future}. An invocation has this form
\texttt{Future(\textit{expr})}. This would evaluate expr in the background. The
call returns immediatly with a result of type \texttt{Future[A]}, being A the type
of expression \texttt{\textit{expr}}.

\subsubsection{Making models reactive by the use of Futures}

Computations that could block the thread of execution, e.g. Blocking I/O like
fetching a resource through the network, should be ran inside Futures. This
contributes to the overall application reactiveness by terms of responsivenes by
improving the overall request latency and by reducing the load of the main
execution thread.

\paragraph{Latency improvements of Futures}

By spawning blocking calls in a separate thread of execution, the response time of
a request transitions from being the sum of the latencies of the independent
blocking operations to the latency of the longest one. This is represented in
figure \ref{fig:futurelatency}.

\begin{figure}[ht]
  \centering
  \includegraphics[width=0.75\textwidth]{futurelatency.png}
  \caption{\label{fig:futurelatency}
    On the left of the figure, a sequential run of blocking operations. On
    the right, a parallel execution which combines operation's results.
  }
\end{figure}

\subsubsection{Futures are highly composable}
Futures are designed to be composable. Instead of transforming the result of the
Future by waiting for the evaluation of the inner expression to be completed,
some methods of a \texttt{Future[A]} allow to transform it by passing a description of what
should be done once the Future is completed. These methods don't block execution until the
Future is completed but return inmediatly a \texttt{Future[B]}, being \texttt{B} the type of
the value obtained by tranforming \texttt{A}.

\paragraph{Freeing the load of the main thread of execution}

In web servers there is usually a fixed thread pool for handling requests. This
thread pool has a maximum number of threads available to prevent the system
from collapsing. In Tomcat,
for example, when this threshold is surpassed, no more
requests can be met and requests start to be stacked \autocite{TheApacheSoftwareFoundation2020ApacheConnector}. As a consequence, latency
starts to increase in a great manner.

Futures have an implicit parameter which defines the execution context in which
the Future expression will be ran \autocite{ScalaScala.concurrent.ExecutionContext}. An
execution context may have associated an specific thread pool. This can be used
for making the blocking calls to be executed in a separate pool for blocking
tasks, freeing the main thread thread pool which is only responsible of spawning the tasks
for Futures, but not of running them.

\subsection{The IO monad as a referential transparent Future}
Futures have many properties that make them an appropiate construct for modeling
asynchronous operations. However, their operations are not referentially transparent.
By the definition of it, if Futures were referentially transparent the result of the operations on a
Future should be the same regardless of the operations being composed
directly or saved in variables.

\lstinputlisting[
caption=An example of how a Future is not referentially transparent.,
float=h,
captionpos=b,
label={lst:impurefuture}
]{ImpureFuture.sc}

This lack of referential transparency is shown in Listing \ref{lst:impurefuture}. Two
Future variables are created on the top level \texttt{futureByVariable} and
\texttt{futureByComposition}. This Futures are created by composing two Futures
created by the same operation, \texttt{random.nextInt}. However, in
\texttt{futureByVariable} as \texttt{Future} caches the results of the function
which was ran on it, the operation isn't performed the second time. This caching
is indeed a side effect given that a variable with the result value has to
be updated once the operation has been performed.

The IO Monad does not perform this caching and it is possible to save a value of type IO into a
variable without losing referential transparency as Listing \ref{lst:pureio} shows.

The project Cats Effect \autocite{CatsHome} provides a standard \texttt{IO} type for
Scala. This \texttt{IO} type provides both synchronous and asynchronous evaluation
models. In addtion, the IO datatype provides an asynchronous processing model similar to
Future but referentially transparent.

\lstinputlisting[
caption=The IO Monad does not cache values and is referentially transparent.,
float=h,
captionpos=b,
label={lst:pureio}
]{PureIO.sc}

\end{document}
