%%% Local Variables:
%%% mode: latex
%%% TeX-master: "main"
%%% End:

\subsection{Computational effects}
In 1991 Eugenio Moggi [[Insert citation]] described a model in category theory
for separating the set of values a object \texttt{A} could have from the notion
of a computation of it, \texttt{T A}, which had an effect on top of this type.
This computation abstracts from the type \texttt{A} the posible results that the
computation could have. Some examples of computations mentioned on the text of
interest to this topics are.

\begin{itemize}
\item Side effects that modify an set of possible states \texttt{S}: \texttt{State[A, S]}
\item Exceptions where \texttt{E} is the set of possible exceptions: \texttt{Either[A, E]}
\item Interactive Input/Output which results in an \texttt{A} value: \texttt{IO[A]}
\end{itemize}

From a programming point of view a very interesting finding of that article is
that this effect over \texttt{A} can be abstracted and define operations that transform
the type A regardless of the effect in which it was contained as long as the
effect \texttt{F} is a monad. %% Is a monad? Has a monad?

\subsubsection{Monads}
A monad is category that provides two operations over a type constructor
\texttt{F[\_]}, \texttt{flatMap}\footnote{On literature is also usually called bind} and
\texttt{pure}\footnote{Also known as point or return}. In the following figure the a
trait that defines the Monad typeclass in Scala is defined.

[[Example]]

The semantics of the operations can be inferred by following the types of them.
The pure function takes a value \texttt{a} and lifts it to the monad context
\texttt{F}. The flatMap operation takes a monad \texttt{F[a]} and a function
which transform an \texttt{a} into a monad with the same context but a possible
different type \texttt{F[B]}. This operation mantains the monad context.

\subsection{Example of monadic effects}
Then some monads instances can be defined for different effects.

\paragraph{Option Monad}
The Option Monad is an effect which may have a value of a type \texttt{A} or
not. An \texttt{Option[A]} has two possible values. \texttt{Some} which means
that the value is present or \texttt{None} meaning that there is no value.

The function flatMap in this case returns a transformed \texttt{Some} if the monad in which
it was applied was also a \texttt{Some} or directly returns a \texttt{None} if
the monad was empty and no transformation could be applied

\paragraph{State Monad}
The State Monad is an effect which given an state of type \texttt{S} performs an state
transition resulting in a value of type \texttt{A}, modeling an state machine.
As state cannot be mutated in place the state transation returns both the new
state and the resulting value. It can be defined this way

When using the state monad flatMap directly returns a monad with the new state
avoiding to explicitly pass the state all along the way.

\paragraph{IO Monad}
Side effects are inherently non functional. A side effect such as reading input
from the keyboard implies that every time that this operation is called the
outcome may be different breaking referential transparency.

However if we look to the description of operations that do side effects they
have nothing that implies a breaking of referential transparency. For example
the function read from keyboard whould have the type \texttt{() => A} which a
pure function could have.

If the side effect is abstracted from the function definition the purity is
mantained. This is what the IO monad does. When a function is passed to the IO
the execution is suspended. It is not executed until the programmer runs the
computation, moment at which purity is lost.

If side effects execution is delayed until the very end of the execution programs
can keep functional at its core and the benefits of purity are not lost until
the interpretation of the side effects.
