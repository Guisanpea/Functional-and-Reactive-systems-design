\documentclass[../main.tex]{subfiles}

\begin{document}

\subsection{Functional systems in Scala}
Scala is not a pure functional programming language. For this reason one has to
be careful when choosing third party tools for creating functional programs.
Some libraries may rely on mutable state or perform side effects in order to
perform their actions, which would break referential transparency.

One first approach to deal with this problem would be to decide having ``mostly
functional'' programs. The problem is that as [[citation]] shows, thinking that
this mixture would give the benefits of functional programs is a
fallacy. Referential transparency is given or not, there is no middle ground. If
some of the elements which compose a program are not referentially transparent
this is propagated to the whole program.

For this reason when choosing libraries it is important to choose foundationally
functional ones. In the case of study presented in this section the Cats library [[citation]]
will be the foundation for building functional programs.

Its documentation describes it as ``a library which provides abstractions for
functional programming in the Scala programming language. ... A broader goal of
Cats is to provide a foundation for an ecosystem of pure, typeful libraries to
support functional programming in Scala applications''.

At its core Cats provides among its abstraction a set of type classes and
datatypes, the majority of this type classes are categories from Category Theory
such as Monad, Monoid or Functor, with instances for many relevant types. But
there are also utility type classes, like the Show typeclass which was shown in
the Section 2.

However Cats doesn't provide any utility for interacting with the real world in
a functional way. There is a separate project, Cats-effect, which ``aims to
provide a standard IO type for the Cats ecosystem, as well as a set
of typeclasses (and associated laws) which characterize general effect types.''.
This IO type was the one presented in section [[Section]] and its data type
provide a set of operations to work with I/O operations such as reading from a
database or communicating through the network.

One element mentioned in the IO section was the fact that running the IO was the
element which made referential transparency lost, and thus it should be the last
operation done in the application. Cats Effect provides \texttt{IOApp} which is
a safe wrapper of a Scala app which expects an \texttt{IO} and runs it. Ensuring
that programs can remain purely functional without exposing any kind of impurity

\section{Serving a Web Server}
On top of Cats-effect, http4s provides a typeful, functional and streamed model
for http servers.

Http4s routes are modeled as a Kleili Category, \mbox{\texttt{Kleisli[OptionT[F,
*], Request, Response]}}, or simplified, \mbox{\texttt{Request => F[Option[Response]]}},
what this mean is that the application routes accept a request and return and
effectful optional response\footnote{In our study case this effect \texttt{F}
  will always be \texttt{IO}}. The return must be effectful because handling the
response may imply performing a side effect such as reading from the database or
accessing a resource from the network, and the response has to be optional
because the given route may not have an appropiate handler.

\end{document}