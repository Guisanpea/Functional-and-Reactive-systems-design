\documentclass[../main.tex]{subfiles}

\begin{document}

\subsection{Functional systems in Scala}
Scala is not a pure functional programming language. For this reason one has to
be careful when choosing third party tools for creating functional programs.
Some libraries may rely on mutable state or perform side effects in order to
perform their actions, which would break referential transparency.

One first approach to deal with this problem would be to decide having ``mostly
functional'' programs. The problem is that this not completely functional
programming model does not work \autocite{MeijerTheWork}. Thinking that
this mixture would give the benefits of functional programs is a
fallacy. Referential transparency is given or not, there is no middle ground. If
some of the elements which compose a program are not referentially transparent,
this is propagated to the whole program.

For this reason, when choosing libraries, it is important to choose foundationally
functional ones. In the case of the study presented in this section, the Cats library
\autocite{Cats:Home} will be the foundation for building functional programs.

Its documentation describes it as ``a library which provides abstractions for
functional programming in the Scala programming language. ... A broader goal of
Cats is to provide a foundation for an ecosystem of pure, typeful libraries to
support functional programming in Scala applications''.

At its core Cats provides, among its abstractions, a set of type classes and
datatypes. The majority of these type classes are categories from Category Theory
such as Monad, Monoid or Functor, with instances for many relevant types. But
there are also utility type classes, like the Show typeclass which was shown in
Section 2.

However, Cats doesn't provide any utility for interacting with the real world in
a functional way. There is a separate project, Cats-effect, 
% pon referencia a Cats-effect
which ``aims to
provide a standard IO type for the Cats ecosystem, as well as a set
of typeclasses (and associated laws) which characterize general effect types.''.
This IO type was the one presented in the monads section and its data type
provide a set of operations to work with I/O operations such as reading from a
database or communicating through the network.

One point mentioned in the IO section was the fact that running an IO computation was the
element which made referential transparency lost, and thus it should be the last
operation done in the application. Cats Effect provides \texttt{IOApp} which is
a safe wrapper of a Scala app which expects an \texttt{IO} and runs it, hence ensuring
that programs can remain purely functional without exposing any kind of impurity.

\subsection{Representing lazy sequences with Streams}
One of the main concerns of the \texttt{IO} datatype is to separate computation
descriptions from their evaluations. They represent a computation to be
made which, when evaluated, will produce a single value. However, there are some
situations in which, rather than obtaining a single value, the programmer may be
interested in receiving an streams of values to be processed. A typical example
of this would be working with I/O sources, like obtaining the lines of a source
file.

This stream of data needs not be consumed entirely. For instance, a function may attempt to
obtain a line of a file matching an specific criteria. As soon as the line is
found, the rest of the file doesn't need to be scanned. Although a function which does
this specific computation can be expressed using \texttt{IO}, the lazy
representation for obtaining the lines from the file can not be expressed by it. For this
reason, the Stream datatype is created.

The datatype which will be used in our study case is the one provided by FS2, 
an acronym for \textbf{Functional Streams for Scala}, and provides ``purely
functional, effectful, and polymorphic stream processing'' \autocite{Fs2:Home}.

A fs2 stream has the following type definition \texttt{Stream[+F[\_], +O]}. In
this definition \texttt{O} is the type of the elements which are contained in
the stream and \texttt{F} is the kind which evaluates the effects that the
descriptions of the values of the streams may have. In our case of study, this
effects will be wrapped by \texttt{IO} if the computations are effectful or by
\texttt{Pure} in case that the operation implies no effect.

For example, the previously mentioned function for getting the first line of a
file matching a criteria can be implemented with FS2 in a functional manner like
shown in Listing \ref{lst:findlinestream}.

\lstinputlisting[
float=h,
caption=Finding the first line matching a criteria with streams.,
captionpos=b,
label={lst:findlinestream}
]{FindLine.sc}

When implementing functional programs, Streams will be a good representation when working with
potentially unbounded, non strict and sequential sources of data.

\subsection{Serving a Web Server}
On top of Cats-effect, http4s provides a typeful, functional and streamed model
for creating http servers.

Http4s routes are modeled as \mbox{\texttt{Kleisli[OptionT[F,
*], Request, Response]}}, or simplifying the Kleisli Category, \mbox{\lstinline{Request => F[Option[Response]]}},
what this mean is that the application routes accept a request and return and
effectful optional response\footnote{In our study case this effect \texttt{F}
  will always be \texttt{IO}}. The result must be effectful because handling the
response may imply performing a side effect, such as reading from the database or
accessing a resource from the network, and the response has to be optional
because the given route may not have an appropiate handler.

\texttt{Request} and \texttt{Response} datatype encode the Http requests and
responses. One interesting element of them is that the body of this request is
represented with the type \texttt{EntityBody[+F[\_]]}. And when looking at its
type definition, the entity body is defined as
\lstinline{type EntityBody[+F[\_]] = Stream[F, Byte]}.
When defining http4s, it was mentioned that it provided a streamed model for
http. This streaming model is very important for reactive applications. If a
request was not streamed, the response time of the web
server would be

\begin{center}
$(\text{latency of receiving the request and processing} +
\text{web server operations latency})$
\end{center}

as the whole request has to be processed on its
entirety before being able to work on it. If the request is streamed, both
the processing of the request and the operations on them can be interleaved,
reducing the overall latency of the request. This can also be applied for the
response, as the response body is also modeled as an \texttt{EntityBody}. By
providing a streamed model of Http, the applications reactivity is increased by
terms of responsiveness.
% de nuevo, a veces http (en minúscula) y otras Http. Hazlo consistente


To create handlers for Http requests, http4s provides the \texttt{HttpRoutes.of[F]} method, which accepts a partial
function \texttt{Request => F[Response[F]]} and lifts it into \texttt{OptionT} type,
being it None in case that the partial function is not defined or Some in case
it is. For example, the routes for the \texttt{/players} resource of the
application is presented in Listing \ref{lst:playerresource}.

\begin{lstlisting}[
  float=h,
  captionpos=b,
  label=lst:playerresource,
  caption={The router for the \texttt{players} resource. Routes implementation is omitted for brevity.}
  ]
  private val routes: HttpRoutes[F] = HttpRoutes.of[F] {
    case GET -> Root =>
      ???
    case GET -> Root / FUUIDVar(id) =>
      ???
    case POST -> Root =>
      ???
  }
\end{lstlisting}

HttpRoutes has a SemigroupK instance so it provides the \texttt{combineK} method
for combining a set of routes into a greater router. Thus, with the SemigroupK
instance in place, an application router can be composed from other routes as Listing
\ref{lst:combineroutes} shows.

\begin{lstlisting}[
  float=h,
  captionpos=b,
  label=lst:combineroutes,
  caption={The router for the whole application.}
  ]
  private def httpApp[F[_]](): Kleisli[F, Request[F], Response[F]] = (
    PlayerController.qualifiedRoutes
    <+>
    MatchController.qualifiedRoutes
  ).orNotFound
\end{lstlisting}

\subsection{Accessing a database}
For working with persistent data within an IO, Doobie, A functional JDBC layer for Scala,
% pon cita a Doobie
provides a monadic API for describing
queries and commands on a database. Operations in Doobie return descriptions of
computations within a context. For example the \texttt{ConnectionIO[A]} type
describes queries and commands that can be expressed when a
\texttt{java.sql.Connection} is available and that returns a value of type
\texttt{A} when interpreted. Internally, ConnectionIO values are represented as a
free monad. This means that ConnectionIO are combinable with high level
operators such as flatMap, or even be foldabled. Listing \ref{lst:queryconnectionio} shows an
example of how to build and combine \texttt{ConnectionIO} values.

\begin{lstlisting}[
  float=h,
  captionpos=b,
  label=lst:queryconnectionio,
  caption={A database query which returns a tuple.}
  ]
  val dbValues: ConnectionIO[(Int, Double)] =
  for {
    a <- sql"select 42".query[Int].unique
    b <- sql"select random()".query[Double].unique
  } yield (a, b)
\end{lstlisting}

Although these values represent queries to a database, they are agnostic about
specific details like how or where they are going to be ran. This is
responsability of the Transactor data type.

\texttt{Transactor} has information about the connection details for the
database but also has information about which execution contexts are being used
for the different operations on the database. One of them is the
\texttt{connectEC} in which threads will be waiting for acquiring a database
connection and also the thread pool in which already connected threads will be
blocked wating for results. This Transactor is parametrized over an effectful
Monad \texttt{F} target.

Once a transactor is available, it can be passed to a ConnectionIO and move the
descriptions to the chosen effectful monad where the description of the
operations to be performed are no longer dependent on the context of
\texttt{java.sql.Connection} but operation descriptions deferred on \texttt{IO}

\subsubsection{Reducing queries mantainance and improving type safety with Quill}
Database schemas evolve over time. Users interests change in the timeline and in
software developed the need to adapt over time is common. It is important to
write code that is able to adapt to these changes and require minimum changes
when evolutions occur.

If queries on the database are done with plain SQL, as changes to the underlying
representation are made, this queries must be rewritten. For this reason Quill
% mete referencia  Quill
provides a Scala case class to Database schema mapping. These queries are
resolved to an internal abstract syntax tree for Quill and translated to a
database query at compile time, having a low runtime overhead. This queries can
even be validated at compile time against the database resulting in a
compilation error in case that the mapping has been incorrectly done.

Quill query model is based in ``a  practical  theory  of  language-integrated
query  based  on quotation and normalisation of quoted terms'' \autocite{CheneyAQuery} and
is able not only to map case classes to database queries but to provide
database queries as for comprehensions and normalization of nested queries to
provide efficient database access for applications.

Queries are type safe as they are performed over a case class but they even have
capabilities such as returning optional values when a value may be missing.

For example, the query shown in Listing \ref{lst:matchquery} extracted for the companion application gets a tennis
match from the database with each player information and points for each set and
even the potential winner of it as an optional value given that it is acquired
as a left join.

\begin{lstlisting}[
float=h,
captionpos=b,
label=lst:matchquery,
caption={A Quill query returning all the elements of a match.}
]
  private def findAllMatches =
    quote {
      val tennisMatchQuery = querySchema[TennisMatchTable]("tennis_match")
      for {
        tMatch <- tennisMatchQuery.sortBy(_.id)
        winner <- query[Player].leftJoin(w => tMatch.winner.contains(w.id))
        player1 <- query[Player].join(_.id == tMatch.player1)
        player2 <- query[Player].join(_.id == tMatch.player2)
        sets <- querySchema[TennisSetTable]("tennis_set")
        .join(_.matchId == tMatch.id)
      } yield (tMatch, winner: Option[Player], player1, player2, sets)
    }
\end{lstlisting}

\subsubsection{Streaming database records}
As presented on the http section, when streaming information, the application
response times are improved. For this reason Doobie provides a way to return
database elements in a streamed manner. This combined with http streaming allows
to improve the overall latency in a great manner.

To return streamed content with Quill, it is only needed to apply the
\texttt{io.getquill.context.stream} function to the quoted query. If combined
with the query of Listing \ref{lst:matchquery}, to stream it, it would only be
needed to do it like \texttt{stream(findAllMatches)}. This query would
nevertheless need to be transacted as any other Doobie request.

\subsection{Making systems reactive with actors}
The main trait which made systems reactive was the use of message passing. One
tool that can be used for modeling message driven architectures is the Actor
model. Its origin dates back to  1973 \autocite{HewittAFormalism}, when they were shown as a way to
model artificial intelligence. At a later moment in 1985 the Actor Model was
presented as a model of concurrency for distributed systems \autocite{Agha1985ACTORS:Systems}.

The
actor model is a great model for efficient distributed systems because it
accomplish the three traits that reactive systems have to meet to be responsive.
Actors, by definition, communicate by means of messages. Actors don't share state nor do they share
code. They consist of an internal state and only communicate through messages.

Actors implementations such as Erlang processes and Akka Actors
% mete cita a Akka
are also resilient.
Supervision Trees is the way of having fault tolerant systems within this
concurrency model. In a Supervision Tree there are actors whose responsability
is to monitor the behaviour of actors in the systems. This supervisors are able to
restart actors and thus gracefully recover from errors in case they occur
\autocite{ErlangBehaviour}, \autocite{SupervisionDocumentation}.

Finally, actors are also elastic because an actor functionality is self
contained. As they receive messages and have a self contained state, an actor can be
multiplied. In case that the actor is not able to cope with the load it is
required to do, more actors can be summoned and the messages can be load balanced
according to the availability of them.

\subsubsection{Making functional actors}
Akka was mentioned before as an implementation of the Actor Model in Scala,
however Akka Actors are slightly composable. Its message passing semantics relies
highly on side effects and its internal state is modeled using unsafe mutable
primitives which make them non referentially transparent.

For this reason, in the scope of functional systems, an actor primitive has been
developed to allow the benefits of asynchronous message passing but in a
referential transparent manner.

To design such actor every element will be analyzed to come up with
an implementation fulfilling the requirements.

Firstly, a mechanism for receiving request and obtaining a
response is needed. It will be called a Mailbox and this will be a type representing a function
\lstinline{mailbox: Input => F[Output]}. As one of the concerns with actors was
composability, this mailbox is highly composable as this function can be wrapped
in a Kleisli Category and as long as \texttt{F} has a monad instance, as is the case of
\texttt{IO}, the Kleisli Category will also have a monad instance.

As multiple request can come to the actor at a given time, a data structure for
storing the multiple messages is needed to hold the incoming messages. For this
purpose fs2 provides a FIFO Queue data structure which holds the messages that
come to the actor.

In this queue, it will not only be saved the input to the actor but also a
\texttt{Deferred} structure. Deferred models a value that will be present at a
later point in time. Its \texttt{get} method will block until an element is set
with its \texttt{complete} method, in our case the actor responsible for
handling the request.

As the element may not be delivered at any point on time,
a timeout can be provided to avoid deadlocks. For this reason the Concurrent
typeclass provides the \texttt{race} method which accepts another effect of the same type
\texttt{F} and returns an \texttt{F[Either[A,B]]} result, being A the type
wrapped in the effect in which race was called and B the type of the effect
which was applied to handle the race. If the former is the first to be completed, the
either will have its value and in the other case it will have the \texttt{B}
value.

To provide a timeout, the deferred can be raced against a timer that is
asleep for the timeout parameter time. If no timeout is provided, the response
will be raced against the receiver thread because if the actor thread is killed
it is guaranteed that the request will never be processed.

With all these elements the Mailbox implementation is presented in Listing
\ref{lst:mailbox}.


\begin{lstlisting}[
float=h,
captionpos=b,
label=lst:matchquery,
caption={A Purely functional mailbox implementation.}
]
  object Mailbox {

    type Mailbox[F[_], Input, Output] = Input => F[Output]

    def apply[Output, Input, F[_] : Concurrent : Monad](
      queue: Queue[F, (Input, Deferred[F, Output])],
      receiver: Fiber[F, Unit],
      timeout: FiniteDuration = 0.seconds
    )(
      implicit timer: Timer[F], F: MonadError[F, Throwable]
    ): Mailbox[F, Input, Output] = (input: Input) => {
      def getTimeout: F[Unit] =
        if (timeout <= 0.seconds) receiver.join
        else timer.sleep(timeout)

      def getTimeoutError: Throwable =
        if (timeout <= 0.seconds) FiberTerminatedException
        else TimeoutException

      for {
        deferredResponse <- Deferred[F, Output]
        _ <- queue.offer1((input, deferredResponse))
        output <- (getTimeout race deferredResponse.get)
        .map {
          case Right(a) => Some(a)
          case _ => F.raiseError(getTimeoutError)
        }
      } yield (output)
    }
  }
\end{lstlisting}

Once the mailbox is available, the next step is to create the actor responsible
of managing the Mailbox. When creating actors the programmer can choose to have
an stateful actor which can have an internal state and produce a response based
on its input and its state while producing a state transition. For managing the
state the \texttt{Ref} datatype \autocite{Cats-effectCats.effect.concurrent.Ref} will be
used, which allows for mutable state.

The actor will initialize its corresponding state, if applicable, and then a queue
for the mailbox will be created. Then the fiber containing the actor receiver function is
created and, lastly, a mailbox with the reference to the actor fiber and queue are
created, being it returned so that the clients can send requests to the actor.
Listings \ref{lst:statefulactor} and {lst:statelessactor} show the
stateful and stateless actor implementation respectively.

\begin{lstlisting}[
float=h,
captionpos=b,
label=lst:statefulactor,
caption={A stateful actor implementation.}
]
def from[
    F[_] : Concurrent : Monad : Timer,
    State,
    Input,
    Output
  ](
    initialState: State,
    receive: (Input, Ref[F, State]) => F[Output]
  ): F[Mailbox[F, Input, Output]] =
    for {
      state <- Ref.of[F, State](initialState)
      queue <- Queue.unbounded[F, (Input, Deferred[F, Output])]
      receiver <- receiver(receive, state, queue)
      mailbox = Mailbox(queue, receiver)
    } yield (mailbox)
\end{lstlisting}

\begin{lstlisting}[
float=h,
captionpos=b,
label=lst:statelessactor,
caption={A stateless actor implementation.}
]
  def from[
    F[_] : Concurrent : Monad : Timer,
    Input,
    Output
  ](
    receive: (Input) => F[Output]
  ): F[Mailbox[F, Input, Output]] =
    for {
      queue <- Queue.unbounded[F, (Input, Deferred[F, Output])]
      receiver <- statelessReceiver(receive, queue)
      mailbox = Mailbox(queue, receiver)
    } yield (mailbox)
\end{lstlisting}

The receiving function for each actor will then:

\begin{enumerate}
\item Dequeue an element of the queue with its corresponding response
  \texttt{Ref} holder. The queue implementation will block until an element is
  available.
\item Apply the receive function that the programmer has provided for the input
  and the corresponding state.
\item Mark the response \texttt{Ref} as completed.
\end{enumerate}

This will be repeated forever in a loop until the fiber associated with the
actor is terminated. Listings \ref{lst:statefulactorreceiver} and
\ref{lst:statelessactorreceiver} show the receive function again for stateful
and stateless actors.

\begin{lstlisting}[
float=h,
captionpos=b,
label=lst:statefulactorreceiver,
caption={The receiving function for a stateful actor.}
]
  private def receiver[Output, Input, State, F[_] : Concurrent : Monad](
    receive: (Input, Ref[F, State]) => F[Output],
    state: Ref[F, State],
    queue: Queue[F, (Input, Deferred[F, Output])]
  ): F[Fiber[F, Unit]] =
    (for {
      (input, response) <- queue.dequeue1
      output <- receive(input, state)
      _ <- response.complete(output)
    } yield ()).foreverM.void.start
\end{lstlisting}

\begin{lstlisting}[
float=h,
captionpos=b,
label=lst:statelessactorreceiver,
caption={The receiving function for a stateless actor.}
]
  private def statelessReceiver[Output, Input, F[_] : Concurrent : Monad](
    receive: (Input) => F[Output],
    queue: Queue[F, (Input, Deferred[F, Output])]
  ): F[Fiber[F, Unit]] =
    (for {
      (input, response) <- queue.dequeue1
      output <- receive(input)
      _ <- response.complete(output)
    } yield ()).foreverM.void.start
\end{lstlisting}

\end{document}