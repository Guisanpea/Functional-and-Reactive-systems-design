\documentclass[../main.tex]{subfiles}

\begin{document}

\subsection{Conclusiones}
After having studied the Functional Programming paradigm the main conclusion is
that it provides a mindset shift when producing programs in terms of their
building blocks.

Imperative programs building blocks are procedures which modify the program
state. The focus of such procedures in the whole program is not \textit{what} the
procedure accomplishes but \textit{how} it does accomplishes it. A procedure in
imperative programming does not have to return a value to do something valuable
to the application. It is so common that the void return type is usually used to
represent this kind of functions that although returning no value provides a
meaningful value to the application.

By contrast, in functional programs the building blocks are functions. This
functions are seen as opaque representations of the value they return. For this
reason programs are rather declarative than imperative. The focus of this
function is \textit{what} they return and not \textit{how} they do it. For this
reason functional programs can be considered as values rather than a set of
instructions that modify the program state.

At the very end functional programs are a description of what operations have to
be done by terms of an \texttt{IO} monad, this has also been seen with the
building blocks of the companion application. For example Doobie functions
returned a \texttt{ConnectionIO} value that was no more than a description that
the value that would be returned once a \texttt{java.sql.Connection} was
provided but didn't do the operations itself. The Http4s routes built a
description of what to with an incoming request but nothing more than that. So
at the end a functional program is an static description of a series of
operations to be performed it is not until the program is interpreted that the
operations finally occur.

This change of paradigm allows the premise proposed in the introduction. As
functions are merely descriptions of the value they produce they can be safely
reused without affecting the already existing application semantics. This enable
faster changes in the functional programs, as the programmer has to be bothered
only about producing new value and not of breaking existing program
functionality.

Apart from that, functional programming enables a new reasoning model for
programs. Given that functions are mere descriptions of the value they produce
the programmer does not have to be judicious on the context in which they are
called. For the programmer it is only relevant the value they produce and thus,
functions are easier to understand. This concept was presented in Chapter 2 as
Local Reasoning.

At the same time the study of reactive systems have given a set of
characteristics to look for in systems to be responsive. Functional Programming
has proven to be a very suitable paradigm for Reactive Applications. The
functional programming constraints make programs more resilient with effects
such as Errors needed to be handled explicitly, reducing the amount of possible
errors.

The functional programming model enables at the same time a safer concurrent
model. With the IO datatype is easy to describe computations that will be
processed in parallel and with functional programs not relying on mutable shared
state some problems such as race conditions and deadlocks are not present.

Finally the immutable message passing semantics of Reactive Systems are very
aligned with the functional style of programming, as functions are no more than
a value produce by terms of its immutable input that can be seen as a message.

All that makes Functional and Reactive systems appropriate for today's
applications which needs not only to be responsive but that need to adapt quick
to change to customers.

\subsection{Future lines of Research}
After this study some concepts have been explored very lightly that could be
interesting to study in depth such as

\paragraph{Category Theory and Functional Programming}

Some categories from category theory have been used to combine values in the
application, such as the Monad and Kleisli categories. An study of the
fundamentals of such categories and their use in functional programming could
help to have more expressive, lawful and concise programs.

\paragraph{Development of a functional Actor System}

An actor implementation based on functional elements have been developed in the
companion application but it didn't have features of more advanced Actor Systems
on other languages such as supervision trees or load balancing which are very
useful have been omitted. Having an implementation which would improve this
elements would make working with actors more convenient

\paragraph{Imperative and Reactive Systems}

In this document the main element of study have been Functional and Reactive
systems. Although some benefits have been found it would be interesting to study
Reactive systems in an imperative paradigm, to compare them with their
functional counterpart to be able to choose the most suitable option for a given
scenario.

\end{document}