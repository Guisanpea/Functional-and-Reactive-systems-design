\documentclass[../main.tex]{subfiles}

\begin{document}

\subsection{Conclusioness}
Tras estudiar los conceptos que fundamentan la Programación Funcional, la principal
conclusión es que ésta implica un cambio de mentalidad a la hora de implementar programas
respecto a los elementos con los que éstos se construyen.

Los elementos con los que se construyen los programas imperativos son procedimientos
que modifican el estado global del programa. Estos procedimientos no se centran exclusivamente
en el valor que producen sino, adenás, en el \textit{cómo} se producen. Un procedimiento en programación
imperativa no tiene porque devolver un resultado para realizar algo constructivo en la aplicación.
Esto es tan habitual que en los lenguajes imperativos existe un tipo, \texttt{void}, que se utiliza
para declarar el tipo devuelto por las ``funciones'' cuyo propósito no es devolver un valor sino  
modificar el estado global del sistema.

Sin embargo, en la programación funcional los bloques elementales son funciones. Estas
funciones pueden ser vistas como representaciones opacas del valor que devuelven. Por eso
mismo los programas funcionales se consideran declarativos más que imperativos ya que
dichas funciones se centran en \textit{qué} valores deben devolver, y no en \texttt{cómo} 
devolverlos. Así, los programas funcionales se pueden interpretar como valores más que como 
una serie de instrucciones que modifican un estado global.
en el valor que producen si no en el \textit{cómo} lo producen. Un procedimiento en programación
imperativa no tiene por que devolver un resultado para realizar algo constructivo en la aplicación.
Tan es así que en los lenguajes imperativos existe un tipo \texttt{void} que puede declararse
como devuelto por las funciones que no devuelven un valor en sí, sino que modifican el estado global
como propósito.

Sin embargo en la programación funcional los bloques elementales son funciones. Estas
funciones se pueden ver como representaciones opacas del valor que devuelven. Por eso
mismo los elementos de programas funcionales se consideran declarativos más que imperativos.
Porque se centran en \textit{qué} valores devuelven, y no en \texttt{cómo} los devuelven.
Por eso mismo los programas funcionales se pueden considerar valores más que una serie de
instrucciones que modifican un estado global.

Al final, los programas funcionales que interactúan con el mundo real son descripciones de
las operaciones a realizar dentro del contexto de una Mónada IO \texttt{IO}. Esto se ha visto
a la hora de crear los bloques que formarían la aplicación que acompaña a este documento. Por ejemplo,
las funciones de Doobie devolvían un valores de tipo \texttt{ConnectionIO} que no eran más que
descripciones de los valores que se devolverían una vez que se proporcione una \texttt{java.sql.Connection}.
Las rutas de Http4s eran una descripción de qué hacer con una petición HTTP entrante por lo que, al
final, los programas funcionales son descripciones estáticas de operaciones a realizar pero
no es hasta que el programa se interpreta que dichas operaciones finalmente acaban realizándose.

Este cambio de paradigma está de acuerdo con las premisas que se proponían en la introducción de este trabajo. 
Como las funciones son meras descripciones del valor que devuelven en función de sus entradas, éstas
pueden ser reusadas sin afectar a otras funciones que ya forman parte dela aplicación. Esto permite
que los cambios de un sistema se puedan realizar de manera más rápida, ya que el programador tiene que 
preocuparse únicamente de producir nuevos valores, y no de si éstos afectan a la funcionalidad ya existente.

Además, la Programación Funcional permite un nuevo modelo de razonamiento sobre los
programas. Dado que las funciones son descripciones de los valores que producen, el
programador no debe pensar en cómo va a afectar el contexto en el que éstas sean utilizadas. Lo 
único que debe preocuparle es el valor que éstas producen y es
por ello que los correspondientes programas son más fáciles de producir y de entender. Este principio es
el que se presentó en el Capítulo 2 como ``Razonamiento Local''.

Al mismo tiempo, el estudio de la programación reactiva ha revelado una serie de
características que los sistemas deben tener para ser capaces de responder eficientemente a 
las interacciones con los usuarios.
La Programación Funcional ha mostrado ser muy apropiada para construir Sistemas Reactivos ya que 
ciertos elementos, como el manejo explícito de los errores, aumenta la resiliencia de
los programas, reduciendo la cantidad de errores en un sistema que pueden acanar siendo no controlados.

La programación funcional también introduce un sistema para la concurrencia muy seguro.
Con el tipo de datos \texttt{IO} es muy fácil representar operaciones que pueden ser
computadas en paralelo y, dado que las distintas partes de un programa funcionale no dependen de un estado
mutable compartido, algunos problemas típicos de los sistemas concurrentes tales como una 
Condición de Secuencia o un Bloqueo Mutuo no puedan ocurrir.

Finalmente, el modelo de intercambio de mensajes de los Sistemas reactivos está bastante
alineado con el estilo de Programación Funcional, dado que los resultados de las funciones
son simplemente valores producidos por entradas inmutables que pueden considerarse mensajes.

Todo ello hace los Sistemas Funcionales y Reactivos apropiados para afrontar las demandas a las que
se enfrentan las aplicaciones de hoy en día, que no solo tienen que responder rápidamente
a las interacciones de los usuarios sino adaptarse constantemente a los cambios que éstos demandan.

\subsection{Líneas Futuras}
Algunos de los conceptos que se han estudiado podrían explorarse con más profundidad. A continuación
se presentan algunos temas que podrían ser estudiados como líneas futuras.


\paragraph{Programación Funcional y Teoría de Categorías}

Algunas categorías de la Teoría de Categorías se han usado para combinar valores
en las aplicaciones, tales como las Mónadas o las Categorías de Kleisli. Un estudio
de los fundamentos de dichas categorías y su relación con la programación funcional
podría ayudar a escribir programas más expresivos, claros y concisos.

\paragraph{Desarrollo de un Sistema Funcional de Actores}

En el documento se ha desarrollado una implementación básica de un actor. Sin embargo,
alguna de las características más avanzadas de algunos Sistemas de Actores, como el balanceo
de carga o los árboles de supervisión, no se han podido implementar debido al tiempo que
requeriría. Una implementación que proporcionase dichas características haría el uso de actores
más conveniente en el contexto de la Programación Funcional.

\paragraph{Sistemas Imperativos y Reactivos}
Este documento se ha centrado en el estudio de Sistemas Funcionales y Reactivos.
Aunque se hayan encontrado beneficios en la combinación de ambos, sería interesante
estudiar los Sistemas Reactivos dentro del paradigma imperativo, para poder compararlos con
sus contrapartes funcionales y así poder elegir la opción más apropiada para cada escenario.

\end{document}