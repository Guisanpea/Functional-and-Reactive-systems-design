\documentclass[../main.tex]{subfiles}

\begin{document}

\subsection{Motivation}
Software systems nowadays are having an unprecedented amount of changing
requirements. As more people have gained access to personal computers, and more
lately mobile devices the amount of users an application could have have
increased greatly. In the past software was developed using a waterfall model,
target users were localized and systems were made to satisfy their needs with
their requirements were known a priori.

This scenario is no longer true. Today's users expect software that adapt to their
needs and in many cases have plenty of alternatives to choose. If an application
is not able to accommodate to the user demands they can browse or
download another one in matter of seconds to minutes at the same time.

With that premise in mind is now more than ever construct applications that are
evolutionary, able to adapt and easily maintainable. This is exactly what
Functional Programming aims to achieve. Functional programs are composed of
functions that are highly composable and that require low maintenance. This
functions are guaranteed to maintain their behavior regardless of external
changes. And as they are free of side effects they provide a seamless
reutilization experience given that they produce no observable result other than
producing the expected use being safe to use regardless of the context in which
they are called.

With that characteristics refactoring is safe because one change of a function
cannot produce a side effect that could affect the functions which are not using
it.

As users demand new features more frequently expecting, no degrading of already
existing behaviour changes are expected to happen occur on code often and having such
properties of functional programs is very desirable in this context.

At the same time, the non functional requirements of software systems have been
growing to be more strict. Nowadays not only human beings consume applications,
the so called Internet of things has been growing massively starting to be
ubiquitously present, from home automation to agriculture there are devices
which connect to the internet to do their work. This can make services to be
consumed by millions of clients simultaneously.

When developing web applications they must be able to scale to this
circumstances and work under high loads without affecting the user experience
and the application responsiveness. The Reactive System term have been then coined to
describe a series of traits and elements that form software applications which
accomplish this characteristics.

\subsection{Objectives}
Firstly, in this document it is going to be exposed what is functional
programming, the characteristics of functional programs and the benefits that
they provide to software systems. At the same times it is going to be showcased
in a practical basis how to create functional programs in the Scala programming
languages and why it is a suitable language for making functional programs.

Secondly it will introduce what are reactive systems, as a principled way to
structure applications that are able to scale and be responsive for users and how to
create them in the context of Functional Programming, combining the benefits of
systems that are able to change and have a good performance

The end goal will be to create a basic set of principles for functional
programming which will allow the programmer not only to recognize what makes
programs functional but to get the most benefit from them. At the same time common
patterns and elements of functional programs will be shown to help to earn the
greatest profit in this paradigm.

Finally a web application made with purely functional constructs will be
developed. This will showcase how Functional Programming can be used to create
applications that interact with the real world and that at the same time have
the reactive traits, thus being both easily adaptable to change and good performant.

\subsection{Structure of the document}
The structure of the document is aligned with the objectives. Chapter 2
will introduce the Functional Programming paradigm along with the principles and
benefits of it.

Chapter 3 will explain then the concept of Reactive Systems, how they relate to
well performing applications and what makes a system reactive.

Both chapters will thus serve as a study of the state of the art and understand the
pillars of both topics to be able to base the rest of the document on this fundamentals.

In Chapter 4 after these fundamentals are known is going to show elements common to
both reactive and functional programs. In it some common structures in
Functional Programming will be presented, at the same time and context there are
also going to be elements of programs that enable reactiveness on an application.

With the knowledge acquired from previous chapters, Chapter 5 is exploring the
companion application to this document, exploring a web application based on
both functional and reactive traits. It will introduce a way of interacting with
the external world in a functional manner and how an application can be
structured based on purely functional constructs.

\subsection{Technologies used}
The whole text will use the Scala programming language for explaining concepts
and to present examples. Ammonite \autocite{Ammonite} will be used as a tool for
creating small Scala scripts as it allows importing third party dependencies and
other scripts in a simple way.

Then the application which will put in practice the concepts exposed previously
will be based on various libraries of the Typelevel ecosystem
\autocite{2020Typelevel.scala} which are based on the Cats library \autocite{Cats:Home}.

\end{document}